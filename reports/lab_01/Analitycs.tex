\newpage
\section{АНАЛИТИЧЕСКАЯ ЧАСТЬ}
\subsection{Описание алгоритмов}

Часто требуется измерить различие между двумя строками (например, в эволюционных, структуральных или функциональных исследованиях биологических строк, в хранении текстовых баз данных, в методах проверки правописания). Есть несколько способов формализации понятия расстояния между строками. Одна общая, и простая, формализация называется редакционным расстоянием; она основана на преобразовании (или редактировании) одной строки в другую серией операций редактирования, выполняемых над отдельными символами. Разрешенные операции редактирования - это \textit{вставка} (insertion) символа в первую строку, \textit{удаление} (deletion) символа из первой строки и \textit{подстановка} или \textit{замена} (substitution или replace) символа из первой строки символом из второй строки. \\
Пусть I - insert, \\
D - delete, \\
R - replace, \\
M - никакая операция. \\
Тогда строка \textit{vintner} может быть доредактирована до \textit{writers} следующим образом: \\
\begin{table}[h]
\begin{center}
\begin{tabular}{ccccccccc}
R & I & M & D & M & D & M & M & I \\
v & & i & n & t & n & e & r & \\
w & r & i & & t & & e & r & s \\
\end{tabular}
\end{center}
\end{table}
\\
\textbf{Определение.} Строка над алфавитом I, D, R, M, которая описывает преобразование одной строки в другую, называется \textit{редакционным предписанием} или, для краткости, \textit{предписанием} этих 2-х строк. \\
\\
\textbf{Определение}. \textit{Редакционное расстояние} между двумя строками определяется как минимальное число редакционных операций - вставок, удалений и подстановок, необходимое для преобразования первой строки во вторую. Совпадения операциями не сичитаются и не засчитываются. \\
\\
\textbf{Задача о редакционном расстоянии} - это задача о вычислении редакционного расстояния между 2-мя данными строками вместе с оптимальным  предписанием, описывающим преобразование, на котором этот минимум достигается. Оптимальное предписание - редакционное предписание, использующее минимальное число редакционных операций. \\
\\
\textbf{Определение.} Для 2-х строк, S$_1$ и S$_2$, значение \textit{D}(\textit{i}, \textit{j}) определяется как редакционное	расстояние между S$_1$[1..\textit{i}] и S$_2$[1..\textit{j}]. \\
\\
\textit{D}(\textit{i}, \textit{j}) - Рекуррентное соотношение. \\

\begin{displaymath}
D[i,j] = \left\{
\begin{array}{ll}
0, & i = 0, j = 0 \\
i, & j = 0, i > 0 \\
j, & j > 0, i < 0 \\
min\left\{
\begin{array}{ll}
D(i,j - 1) + 1, \\
D(i - 1,j) + 1, \\
D(i - 1,j - 1) + t(S_1[i],S_2[j])
\end{array}
\right.
& j > 0, i > 0
\end{array}
\right.
\end{displaymath}
\\
где

\begin{displaymath}
t(S_1[i],S_2[j]) = \left\{
\begin{array}{ll}
0, & S_1[i] = S_2[j] \\
1 & \text{otherwise}
\end{array}
\right.
\end{displaymath}