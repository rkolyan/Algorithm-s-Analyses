\newpage
\section{ЗАКЛЮЧЕНИЕ}

В данной лабораторной работе был реализован алгоритм Левенштейна, позволяющий решать множество прикладных задач:автоматического исправления ошибок в слове, сравнения файлов, а в биоинформатике генов и хромосом. Проведено сравнение 3-х реализаций алгоритмов, выявлены их слабые места. Алгоритм с рекурсией является самым медленным, его стоит заменить базовым или модифицированным. Базовый и модифицированный сильно по скорости в данной реализации не различаются.