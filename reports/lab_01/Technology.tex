\newpage
\section{ТЕХНОЛОГИЧЕСКАЯ ЧАСТЬ}
\subsection{Требования к программному обеспечению}

\begin{flushleft}
Программа должна работать на операционной системе Arch Linux. Программа должна
содержать 2 режима:
\begin{itemize}
\item Пользовательский
\item Экспериментальный
\end{itemize}
В пользовательском режиме пользователь должен иметь возможность вводить строки и получать результат работы 3-х алгоритмов. В экспериментальном режиме засекается процессорное время работы каждого алгоритма, результаты записываются в отдельные файлы. Впоследствии данные из этих файлов можно вывести в виде графика.
\end{flushleft}

\newpage
\subsection{Средства реализации}
Для реализации данных алгоритмов был выбран язык программирования С, компилятор
gcc и некоторые функции из библиотеки glibc (memcpy, malloc и тд...). \\
Удобства, предоставляемые языком C:
\begin{itemize}
\item Прямой доступ к памяти;
\item Возможность составлять простейшие структуры данных;
\end{itemize}
Для вывода графиков использовался Python3 (библиотека Matplotlib)

\newpage
\subsection{Листинг кода}
\lstdefinestyle{customc}{
  belowcaptionskip=1\baselineskip,
  breaklines=true,
  frame=L,
  xleftmargin=\parindent,
  language=C,
  showstringspaces=false,
  basicstyle=\footnotesize\ttfamily
}

\lstinputlisting[captionpos=b, caption=\label{listings:listing1}Стандартная реализация алгоритма Левенштейна(\ref{images:levenstein}), style=customc]{listing1.c}
\lstinputlisting[captionpos=b, caption=\label{listings:listing2}Рекурсивная реализация алгоритма Левенштейна(\ref{images:recursive_levenstein}), style=customc]{listing2.c}
\lstinputlisting[captionpos=b, caption=\label{listings:listing3}Стандартная реализация алгоритма Дамерау-Левенштейна(\ref{images:damerau_levenstein}), style=customc]{listing3.c}