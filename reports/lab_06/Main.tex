\documentclass[a4paper,12pt]{article}
\usepackage[T2A]{fontenc}
\usepackage[utf8x]{inputenc}
\usepackage[english,russian]{babel}
\usepackage{amssymb,amsfonts,amsmath,mathtext}
\usepackage[unicode]{hyperref}
\usepackage{listings}
\usepackage{graphicx}
\usepackage{float}
\graphicspath{{images/}}
\newcommand{\anonsection}[1]{\section*{#1}\addcontentsline{toc}{section}{#1}}

\begin{document}

% Титульный лист

\begin{titlepage}
\newpage

\begin{center}

\textit{Министерство науки и высшего образования Российской Федерации \\ 
Федеральное государственное бюджетное образовательное \\
учреждение высшего образования \\
«Московский государственный технический университет \\
имени Н.Э. Баумана (национальный исследовательский университет)» \\
(МГТУ им. Н.Э. Баумана) \\}
\hrulefill
\end{center}

\vspace{2em}

\begin{flushleft}
ФАКУЛЬТЕТ <<Информатика и системы управления>> \\
\vspace{0.5em}
КАФЕДРА <<Программное обеспечение ЭВМ и информационные технологии>>
\end{flushleft}


\vspace{8em}

\begin{center}
\LARGE Лабораторная работа №3 \\
\end{center}

\vspace{1.5em}

\begin{center}
\textsc{Сортировки массивов}
\end{center}

\vspace{6em}

\begin{center}
Головнев Н.В.

\vspace{4em}

ИУ7-54Б
\end{center}

\vspace{\fill}

\begin{center}
Москва 2019
\end{center}

\end{titlepage}

\tableofcontents

% Введение

\newpage
\anonsection{ВВЕДЕНИЕ}

\newpage
\anonsection{ПОСТАНОВКА ЗАДАЧИ}
Цель задачи: Изучить муравьиный алгоритм на материале решения задачи коммивояжера.\\
Задача:\\
\begin{itemize}
\item Описать методы полного перебора и эвристический, основанный на муравьином алгоритме;
\item Реализовать эти методы;
\item Выбрать класс данных, подготовить данные ;
\item Провести параметризацию метода, основанного на муравьином алгоритме;
\item Интерпретировать результаты и сравнить их с результатами метода полного перебора.
\end{itemize}

\newpage
\section{АНАЛИТИЧЕСКАЯ ЧАСТЬ}
\subsection{Описание алгоритма}
Формулировка задачи коммивояжера:\\
Дано $N$ узлов, расположенных на плоскости. Задан входной узел (Вх) и выходной узел (Вых). Необходимо обнаружить кратчайший путь, охватывающий все узлы, начинающийся во входном узле, заканчивающийся в выходном узле и проходящий через каждый узел только 1 раз.\\
\begin{center}
\textbf{Метод полного перебора}
\end{center}
Полный перебор — метод решения задачи путем перебора всех возможных вариантов. Сложность полного перебора зависит от количества всех возможных решений задачи. Если пространство решений очень велико, то полный перебор может не дать результатов в течение нескольких лет или даже столетий.\\

\begin{center}
\textbf{Муравьиный алгоритм}
\end{center}
Муравьиный алгоритм - один из эффективных полиномиальных алгоритмов (эвристический) для нахождения приближённых решений задачи коммивояжёра, а также решения аналогичных задач поиска маршрутов на графах. Суть подхода заключается в анализе и использовании модели поведения муравьёв, ищущих пути от колонии к источнику питания.\\
Пусть города заданы матрицей кратчайших расстояний.\\
У муравья присутствуют 3 чувства:
\begin{enumerate}
\item Зрение (определяет длину ребра)
\item Обоняние (чует феромон)
\item Память (запоминает пройденный маршрут)
\end{enumerate}
Пусть феромон на ребре $D_{ij}$ в момент времени $t = \tau_{ij}$. На старте инициализируется матрица $\tau$ константами.\\
Находясь в городе $i$ муравей $k$  принимает решение о выборе случайного города $j$ по следующему правилу:\\
\begin{equation}\label{equations:equation1}
P_{k,ij} = \left\{
\begin{array}{ll}
\frac{\tau_{ij}(t)^\alpha * (\eta_{ij})^\beta}{\sum_{q \in C}(\tau_{iq}(t)^\alpha * (\eta_{iq})^\beta)} & j \in C \\
0 & j \in C \\
\end{array}
\right.
\end{equation}
Здесь:\\
$P_{k,ij}$ - вероятность того, что муравей $k$ будет двигаться из города $i$ в город $j$,\\
$C$ - множество городов, которые муравей ещё не посещал,\\
$\tau_{ij}$ - кол-во феромонов на ребре $ij$,\\
$\alpha$ - параметр, контролирующий влияние $\tau_{ij}$,\\
$\eta_{ij}$ - привлекательность ребра ($\eta_{ij} = \frac{1}{D_{ij}}$),\\
$\beta$ - параметр, контролирующий влияние $\eta_{ij}$.\\
Обновление феромонов:\\
\begin{equation}\label{equations:equation2}
\tau_{ij} = (1 - \rho)\tau_{ij} + \Delta\tau_{ij}
\end{equation}
Где:\\
$\tau_{ij}$ - воличество феромона на ребре $ij$,\\
$\rho$ - скорость испарения феромона,\\
$\Delta\tau_{ij}$ - количество отложенного муравьем феромона, обычно определяется как:\\
\begin{equation}\label{equations:equation3}
\Delta\tau_{ij}^k = \left\{
\begin{array}{ll}
\frac{Q}{L_k} & \text{if ant } k \text{ travels on edge } ij\\
0 & otherwise 
\end{array}
\right.
\end{equation}
$L_k$ - стоимость пути, пройденного $k$-м муравьем, $Q$ - параметр, имеющий значение порядка длины оптимального пути.\\
Псевдокод решения задачи коммивояжёра при помощи муравьиного алгоритма:\\
1. Ввод матрицы расстояний $D$\\
2. Инициализация параметров алгоритма – $Q$,$\beta$,$\alpha$\\
3. Инициализация рёбер – присвоение видимости $\eta_ij$ и начальной концентрации феромона \\
4. Размещение муравьёв в случайно выбранные города без совпадений \\
5. Выбор начального кратчайшего маршрута\\
6. Цикл по времени жизни колонии $t \leftarrow 1$ to $t_{max}$ \\
7.  Цикл по всем муравьям $k \leftarrow 1$ to $m$\\
8.  Построить маршрут $T_k(t)$ по правилу \eqref{equations:equation1} и рассчитать длину $L_k(t)$\\
9.  конец цикла по муравьям \\
10.  Проверка всех на лучшее решение по сравнению с $L^*$\\
11.  Вслучае если решение $L_k(t)$ лучше, обновить $L^*$ и $T^*$\\
12.  Цикл по всем рёбрам графа \\
13.  Обновить следы феромона на ребре по правилам \eqref{equations:equation2} и \eqref{equations:equation3}\\ 
14.  конец цикла по рёбрам\\
15. конец цикла по времени\\
16. Вывести кратчайший маршрут $T^*$  и его длину $L^*$\\

\newpage
\subsection{Вывод}
В данном разделе были описаны принципы работы алгоритмов, решающие задачу коммивояжера - муравьиный алгоритм и алгоритм, использующий полный перебор. На основе этих принципов теперь можно написать приложение, реализующее поиск оптимального расстояния для обхода всех городов.
% Конструкторская часть

\newpage
\section{КОНСТРУКТОРСКАЯ ЧАСТЬ}

\subsection{Разработка алгоритма}
На вход у алгоритма передаются в качестве параметров:
%\begin{enumerate}
%TODO
%\end{enumerate}
Возвращаемое значение: код ошибки (0 в случае успеха, иначе отрицательное значение). \\
Побочные эффекты:

\newpage
\subsection{Схема алгоритма}
Ниже представлена схема работы муравьиного алгоритма (ACO):\\
%TODO Сюда нужно вставить схему

\newpage
\subsection{Вывод}
На основе аналитических данных были разработаны требования к разрабатываемому алгоритму, а также приблизительная схема работы алгоритма.


\newpage
\section{ТЕХНОЛОГИЧЕСКАЯ ЧАСТЬ}
\subsection{Требования к программному обеспечению}

Программа должна работать на операционной системе Arch Linux.
%TODO

\newpage
\subsection{Средства реализации}
Для реализации данных алгоритмов был выбран язык программирования С, компилятор gcc и некоторые функции из библиотеки glibc (memcpy, malloc и тд...). \\

\newpage
\subsection{Листинг кода}
Ниже приведены реализации алгоритмов на С.\\
\lstdefinestyle{customc}{
  belowcaptionskip=1\baselineskip,
  breaklines=true,
  frame=L,
  xleftmargin=\parindent,
  language=C,
  showstringspaces=false,
  basicstyle=\footnotesize\ttfamily
}

%TODO Добавить листинг
%\lstinputlisting[captionpos=b, caption=\label{listings:listing2}Муравьиный алгоритм(\ref{images:scheme}), style=customc]{listing.c}

\newpage
\subsection{Вывод}
Используя язык программирования C, в ходе практической работы была спроектирована и написана реализация муравьиного алгоритма (ACO).

\newpage
\section{ЭКСПЕРИМЕНТАЛЬНАЯ ЧАСТЬ}
\subsection{Характеристики аппаратного и программного обеспечения}
% Часть которую никогда нельзя менять
Тестирование приложения проводилось на машине со следующими характеристиками:\\
\begin{itemize}
\item Процессор Intel® Core™ i7-7700HQ;
\item Оперативная память 16 ГБ;
\item Операционная система - Arch Linux с рабочим окружением Cinnamon.
\end{itemize}

\newpage
\subsection{Примеры работы}
На Рис. \ref{images:example}, предсавленном ниже, демонстрируется работа приложения. Запуск приложения осуществляется из эмулятора терминала в Arch Linux.
%TODO Что запрашивает на вход, что выдает на выходе и тд...
%TODO Добавить пример работы приложения
%\begin{figure}[h]
%\center{\includegraphics[scale=0.5]%{example.png}}
%\caption{Пример работы приложения}
%\label{images:example}
%\end{figure}

%TODO Подумай о тестриовании, нужно ли это вообще

\newpage
\subsection{Оценка затрачиваемой памяти}
%TODO Нужно провести оценку затрачиваемой памяти

\newpage
\subsection{Вывод}
%TODO 

\newpage
\anonsection{ЗАКЛЮЧЕНИЕ}
%TODO Напиши и это пожалуйста

\newpage
\anonsection{СПИСОК ИСТОЧНИКОВ}
\begin{enumerate}
\item Чураков Михаил, Якушев Андрей, 2006, Муравьиные алгоритмы.
\item Штовба С. Д. Муравьиные алгоритмы, Exponenta Pro. Математика в приложениях. 2004. № 4
\item Хабр, муравьиные алгоритмы - \url{https://habr.com/ru/post/105302/}
\end{enumerate}

\end{document}
