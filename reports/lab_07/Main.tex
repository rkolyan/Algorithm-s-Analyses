\documentclass[a4paper,12pt]{article}
\usepackage[T2A]{fontenc}
\usepackage[utf8x]{inputenc}
\usepackage[english,russian]{babel}
\usepackage{amssymb,amsfonts,amsmath,mathtext}
\usepackage[unicode]{hyperref}
\usepackage{listings}
\usepackage{graphicx}
\usepackage{float}
\graphicspath{{images/}}
\newcommand{\anonsection}[1]{\section*{#1}\addcontentsline{toc}{section}{#1}}

\begin{document}

% Титульный лист

\begin{titlepage}
\newpage

\begin{center}

\textit{Министерство науки и высшего образования Российской Федерации \\ 
Федеральное государственное бюджетное образовательное \\
учреждение высшего образования \\
«Московский государственный технический университет \\
имени Н.Э. Баумана (национальный исследовательский университет)» \\
(МГТУ им. Н.Э. Баумана) \\}
\hrulefill
\end{center}

\vspace{2em}

\begin{flushleft}
ФАКУЛЬТЕТ <<Информатика и системы управления>> \\
\vspace{0.5em}
КАФЕДРА <<Программное обеспечение ЭВМ и информационные технологии>>
\end{flushleft}


\vspace{8em}

\begin{center}
\LARGE Лабораторная работа №8 \\
\end{center}

\vspace{1.5em}

\begin{center}
\textsc{Поиск в словаре}
\end{center}

\vspace{6em}

\begin{center}
Головнев Н.В.

\vspace{4em}

ИУ7-54Б
\end{center}

\vspace{\fill}

\begin{center}
Москва 2019
\end{center}

\end{titlepage}

\tableofcontents

% Введение

\newpage
\anonsection{ВВЕДЕНИЕ}
% Напиши введение

\newpage
\anonsection{ПОСТАНОВКА ЗАДАЧИ}
% Напиши постановку задачи

\newpage
\section{АНАЛИТИЧЕСКАЯ ЧАСТЬ}
\subsection{Описание алгоритма}
% Опиши здесь все алгоритмы с картинками

\newpage
\subsection{Вывод}
% Напиши вывод



\newpage
\section{КОНСТРУКТОРСКАЯ ЧАСТЬ}

\subsection{Разработка алгоритма}
На вход у всех алгоритмов передаются в качестве параметров:
\begin{enumerate}
\item Строка, в которой происходит поиск строки-образца;
\item Длина этой строки; 
\item Строка-образец;
\item Длина строки-образца;
\item Дополнительная память (для обычного алгоритма поиска она не нужна);
\item Ссылка или указатель на переменную, в которую записывается результат.
\end{enumerate}
Возвращаемое значение: код ошибки (0 в случае успеха, иначе отрицательное значение). \\
Побочные эффекты: изменяется значение переменной результата.

\newpage
\subsection{Схемы алгоритмов}
Ниже представлены схемы алгоритмов поиска в образца в тексте.
% Вставь сюда эти гребаные схемы

\newpage
\subsection{Вывод}
На основе аналитических данных были разработаны требования к разрабатываемым алгоритмам.


\newpage
\section{ТЕХНОЛОГИЧЕСКАЯ ЧАСТЬ}
\subsection{Требования к программному обеспечению}
Программа должна работать на операционной системе Arch Linux. 
Программа должна запускаться из консоли (или терминала) следующей командой:\\
\textit{./app.exe <haystack> <needle>} \\
\textit{app.exe} - само приложение. \textit{<haystack>} - строка текста, в которой будет проводиться поиск строки-образца. \textit{<needle>} - строка-образец.
На выход программа должна печатать позиции, в которых подстрока была обнаружена в строке, а если таких позиций нет, то печатать 0.

\newpage
\subsection{Средства реализации}
Для реализации данных алгоритмов был выбран язык программирования С, компилятор gcc и некоторые функции из библиотеки glibc (memcpy, malloc и тд...). \\

\newpage
\subsection{Листинг кода}
Ниже приведена реализация алгоритма на С.\\
\lstdefinestyle{customc}{
  belowcaptionskip=1\baselineskip,
  breaklines=true,
  frame=L,
  xleftmargin=\parindent,
  language=C,
  showstringspaces=false,
  basicstyle=\footnotesize\ttfamily
}
\lstinputlisting[captionpos=b, caption=\label{listings:listing1}Реализация алгоритма поиска в словаре(\ref{images:scheme1}), style=customc]{listing1.c}
\lstinputlisting[captionpos=b, caption=\label{listings:listing2}Структура узла дерева, style=customc]{listing2.c}
\newpage
\subsection{Вывод}
Используя язык программирования C, в ходе практической работы был спроектирован и написан алгоритм поиска в словаре.

\newpage
\section{ЭКСПЕРИМЕНТАЛЬНАЯ ЧАСТЬ}
\subsection{Характеристики аппаратного и программного обеспечения}
% Часть которую никогда нельзя менять
Тестирование приложения проводилось на машине со следующими характеристиками:\\
\begin{itemize}
\item Процессор Intel® Core™ i7-7700HQ;
\item Оперативная память 16 ГБ;
\item Операционная система - Arch Linux с рабочим окружением Cinnamon.
\end{itemize}

\newpage
\subsection{Примеры работы}
На Рис. \ref{images:example}, предсавленном ниже, демонстрируется работа приложения. Запуск приложения осуществляется из эмулятора терминала в Arch Linux.
\begin{figure}[h]
\center{\includegraphics[scale=0.5]{example.png}}
\caption{Пример работы приложения}
\label{images:example}
\end{figure}

\newpage
\subsection{Оценка затрачиваемой памяти}
Размер памяти (в байтах) $M$, выделенной под словарь можно вычислить по формуле:\\
\begin{equation}
M =  \sum_{i = 0}^{\vert V \vert} T_i + P * \vert V \vert
\end{equation}
где:\\
\begin{equation}
T_i = \left\{
\begin{array}{ll}
\vert V \vert * \alpha + \sum_{j = 0}^N T_{ij} + \beta + P\text{,} & \text{if } T_i \text{ exists} \\
0\text{,} & \text{if } T_i \text{ not exists} \\
\end{array}
\right.
\end{equation}
$V$ - заданный алфавит, \\
$\alpha$ - размер символа в байтах, \\
$\beta$ - размер переменной счетчика(если тип integer - 4 байт),\\
$N$ - кол-во дочерних узлов дерева словаря (значение счетчика), \\
$P$ - память, выделяемая под указатели.

\newpage
\subsection{Вывод}
В ходе эксперимантольной части работы было протестирована работа приложения. Была проведенка оценка затрачиваемой памяти этого алгоритма. 

\newpage
\anonsection{ЗАКЛЮЧЕНИЕ}
Поиск в словаре в основном нужен для проверки орфографии в редакторах текста. Также, он ускоряет написание программ в интегрированных средствах разработки путем автоподставления слов (переменных, функций и тд).

\newpage
\anonsection{СПИСОК ИСТОЧНИКОВ}
\begin{itemize}
\item \label{site:wikipedia}https://ru.wikipedia.org/wiki/
\item \label{site:habr}https://habr.com/ru/post/422085/
\end{itemize}

\end{document}
